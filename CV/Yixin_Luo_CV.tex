%%%%%%%%%%%%%%%%%%%%%%%%%%%%%%%%%%%%%%%%%
% "ModernCV" CV and Cover Letter
% LaTeX Template
% Version 1.3 (29/10/16)
%
% This template has been downloaded from:
% http://www.LaTeXTemplates.com
%
% Original author:
% Xavier Danaux (xdanaux@gmail.com) with modifications by:
% Vel (vel@latextemplates.com)
%
% License:
% CC BY-NC-SA 3.0 (http://creativecommons.org/licenses/by-nc-sa/3.0/)
%
% Important note:
% This template requires the moderncv.cls and .sty files to be in the same 
% directory as this .tex file. These files provide the resume style and themes 
% used for structuring the document.
%
%%%%%%%%%%%%%%%%%%%%%%%%%%%%%%%%%%%%%%%%%

%----------------------------------------------------------------------------------------
%	PACKAGES AND OTHER DOCUMENT CONFIGURATIONS
%----------------------------------------------------------------------------------------

\documentclass[12pt,letterpaper,sans]{moderncv} % Font sizes: 10, 11, or 12; paper sizes: a4paper, letterpaper, a5paper, legalpaper, executivepaper or landscape; font families: sans or roman

\moderncvstyle{classic} % CV theme - options include: 'casual' (default), 'classic', 'oldstyle' and 'banking'
\moderncvcolor{blue} % CV color - options include: 'blue' (default), 'orange', 'green', 'red', 'purple', 'grey' and 'black'

\usepackage{lipsum} % Used for inserting dummy 'Lorem ipsum' text into the template

\usepackage[scale=0.85]{geometry} % Reduce document margins
%\setlength{\hintscolumnwidth}{3cm} % Uncomment to change the width of the dates column
%\setlength{\makecvtitlenamewidth}{10cm} % For the 'classic' style, uncomment to adjust the width of the space allocated to your name

\usepackage[numbers,sort&compress]{natbib}

\newcommand*{\footersymbol}{%
    {~~~{\rmfamily\textbullet}~~~}}


\newif\ifcameraready%
\camerareadytrue%
% \camerareadyfalse%

%----------------------------------------------------------------------------------------
%	NAME AND CONTACT INFORMATION SECTION
%----------------------------------------------------------------------------------------

\firstname{Yixin} % Your first name
\familyname{Luo} % Your last name

% All information in this block is optional, comment out any lines you don't need
\title{Curriculum Vitae}
% \address{123 Broadway}{City, State 12345}
\mobile{(734) 546 7629}
% \phone{(000) 111 1112}
% \fax{(000) 111 1113}
\email{yixinluo@cs.cmu.edu}
\homepage{www.yixinluo.com}{www.yixinluo.com} % The first argument is the url for the clickable link, the second argument is the url displayed in the template - this allows special characters to be displayed such as the tilde in this example
\extrainfo{Google Scholar: \href{https://scholar.google.com/citations?user=7IbFEaMAAAAJ}{Yixin Luo}
% \footersymbol%
\\
GitHub: \href{https://github.com/camellyx}{camellyx}
% \footersymbol%
\\
LinkedIn: \href{https://www.linkedin.com/in/luoyixin}{luoyixin}}

% TODO: shrink photo size to fit more on the first page
\photo[80pt][0.4pt]{pictures/yixinluo} % The first bracket is the picture height, the second is the thickness of the frame around the picture (0pt for no frame)
% \quote{``A witty and playful quotation'' --- John Smith}

%----------------------------------------------------------------------------------------

\begin{document}

%----------------------------------------------------------------------------------------
%	COVER LETTER
%----------------------------------------------------------------------------------------

% To remove the cover letter, comment out this entire block

% \clearpage

% \recipient{HR Department}{Corporation\\123 Pleasant Lane\\12345 City, State} % Letter recipient
% \date{\today} % Letter date
% \opening{Dear Sir or Madam,} % Opening greeting
% \closing{Sincerely yours,} % Closing phrase
% \enclosure[Attached]{curriculum vit\ae{}} % List of enclosed documents

% \makelettertitle % Print letter title

% \lipsum[1-2] % Dummy text
% \lipsum[4] % Dummy text

% \makeletterclosing % Print letter signature

% \newpage

%----------------------------------------------------------------------------------------
%	CURRICULUM VITAE
%----------------------------------------------------------------------------------------

\makecvtitle% Print the CV title

%----------------------------------------------------------------------------------------
%	EDUCATION SECTION
%----------------------------------------------------------------------------------------

\section{Education}

\cventry{2012.5--2018.3}{Ph.D. in Computer Science}{Carnegie Mellon University}{Pittsburgh, PA}{}
{PhD Thesis: \href{https://arxiv.org/abs/1808.04016}
{``Architectural Techniques for Improving NAND Flash Memory Reliability''},\\
advised by Prof.\ \textbf{\href{http://users.ece.cmu.edu/~omutlu/}{Onur Mutlu}.}
% \newline{}
% Selected Coursework: Deep Learning, Advanced Database Systems, Deep Reinforcement Learning, Machine Learning, Optimizing Compilers, Operating Systems and Distributed Systems, Advanced Cloud Computing, Graduate Algorithms, Computer Architecture, Computer Networks
}  % Arguments not required can be left empty
\cventry{2010.9--2012.5}{B.S. in Computer Engineering}{University of Michigan}{Ann Arbor, MI}{}{GPA\@: 3.9/4.0. Dean's List 2010, 2011, EECS Scholar 2010.}
\cventry{2008.9--2012.8}{B.S. in Electrical and Computer Engineering}{Shanghai Jiao Tong University}{China}{}{GPA\@: 3.8/4.0. Dean's List 2009.}

% \section{Masters Thesis}

% \cvitem{Title}{\emph{Money Is The Root Of All Evil --- Or Is It?}}
% \cvitem{Supervisors}{Professor James Smith \& Associate Professor Jane Smith}
% \cvitem{Description}{This thesis explored the idea that money has been the cause of untold anguish and suffering in the world. I found that it has, in fact, not.}

%----------------------------------------------------------------------------------------
%	WORK EXPERIENCE SECTION
%----------------------------------------------------------------------------------------

\section{Experience}

% \subsection{Vocational}

\cventry{2018.8--present}{Software Engineer}{Google Inc.}{Sunnyvale, CA}{}{
Working with Dr. \textbf{\href{https://www.linkedin.com/in/jichuan}{Jichuan Chang}} on improving storage and database platforms performance.
}

\cventry{2018.4--2018.8}{Postdoctoral Researcher in ECE}{Carnegie Mellon University}{Pittsburgh, PA}{}{
Worked with Dr.\ \textbf{\href{http://users.ece.cmu.edu/~saugatag/}{Saugata Ghose}} on developing new techniques for improving SSD storage system reliability.
}


\cventry{2015.5--2015.8, 2016.5--2016.8}{Engineering Intern}{Seagate Technology}{Lakeview, CA}{}{
Worked with Dr.\ \textbf{\href{https://www.linkedin.com/in/erich-haratsch-134349/}{Erich Haratsch}} on developing new SSD controller algorithms for next-generation NAND flash memories.
\newline{}
Detailed achievements:
\begin{itemize}
\item Developed 10 new techniques and 4 new models to improve SSD lifetime by up to 12.9$\times$
\item Developed new tools to automatically test and analyze seven types of SSD errors
\item Collected and analyzed 700~GB of real SSD error data using machine learning and statistical modeling techniques
\end{itemize}}

%------------------------------------------------

\cventry{2013.6--2013.8}{Research Intern}{Microsoft Research}{Redmond, WA}{}{
Worked with Dr.\ \textbf{\href{https://www.microsoft.com/en-us/research/people/liuj/}{Jie Liu}} on developing new server architectures to tolerate memory errors in large-scale data centers.
\newline{}
Detailed achievements:
\begin{itemize}
\item Developed a new server architecture to reduce data center TCO by 2.7\%
\item Characterized memory error vulnerability of 3 important production data-intensive applications running in Microsoft data centers
\end{itemize}}

%----------------------------------------------------------------------------------------
%	AWARDS SECTION
%----------------------------------------------------------------------------------------

\section{Awards}

\cvitem{2017}{\textbf{DFRWS EU Best Paper Award}}
\cvitem{2015}{\textbf{HPCA Best Paper Runner Up}}
\cvitem{2012}{\textbf{HPCA Best Paper Award}}

%----------------------------------------------------------------------------------------
%   PUBLICATIONS SECTION
%----------------------------------------------------------------------------------------

\setlength{\bibsep}{8pt plus 0.3ex}
\bibliographystyle{unsrtnat}
\bibliography{publications}
\nocite{*}

%----------------------------------------------------------------------------------------
%   CONFERENCE TALKS SECTION
%----------------------------------------------------------------------------------------

\section{Conference Talks}

\cvlistitem{\textbf{Architectural Techniques for Improving NAND Flash Memory Reliability}, \newline
            Thesis Defense: CMU 2018, \newline
            Job Talk: Alibaba, Baidu, Seagate, 2018, \newline
            Seminar: Tsinghua University, CAS-ICT, SJTU, Kyushu University, NTU Taipei, NCTU, TSMC, City University of Hong Kong, Seoul National University, KAIST, Microsoft Research, 2018.}
\cvlistitem{\textbf{Improving 3D NAND Flash Memory Lifetime by Tolerating Early Retention Loss and Process Variation}, Sigmetrics 2018, PDL Retreat 2017.}
\cvlistitem{\textbf{HeatWatch: Improving 3D NAND Flash Memory Device Reliability by Exploiting Self-Recovery and Temperature Awareness}, HPCA 2018, PDL Retreat 2017.
  \newline \href{https://youtu.be/7ZpGozzEVpY}{(Lightning session talk on YouTube: https://youtu.be/7ZpGozzEVpY)}}
\cvlistitem{\textbf{Improving SSD Lifetime with Access Pattern and Flash Device Awareness}, Google PhD Summit 2017.}
% \cvlistitem{\textbf{Mitigating Data Retention and Process Variation Errors in 3D NAND Based SSDs}, PDL Retreat 2017.}
\cvlistitem{\textbf{Online Flash Channel Modeling and Its Applications}, Flash Memory Summit, PDL Retreat, 2016.}
% \cvlistitem{\textbf{Data Retention in Flash-Based SSDs}, CMU SCS Student Seminar Series 2016.}
% \cvlistitem{\textbf{Data Retention and Read Disturb in MLC NAND Flash Memory}, PDL Retreat 2016.}
\cvlistitem{\textbf{Data Retention in MLC NAND Flash Memory}, Flash Memory Summit, PDL Retreat, CMU SCS Student Seminar Series, 2015.}
\cvlistitem{\textbf{WARM\@: Write-hotness Aware Retention Management}, MSST 2015.}
\cvlistitem{\textbf{Read Disturb Errors in MLC NAND Flash Memory}, DSN, PDL Retreat, 2015.}
\cvlistitem{\textbf{Data Retention in MLC NAND Flash Memory} \textbf{\textit{(Best paper session)}}, HPCA, PDL Retreat, CALCM Seminar, 2015.}
\cvlistitem{\textbf{Optimizing Data Center Cost via Heterogeneous Reliability Memory}, DSN, PDL Retreat, 2014.}

% \cvitemwithcomment{2018}{HeatWatch: Improving 3D NAND Flash Memory Device Reliability}{HPCA}
% \cvitemwithcomment{}{by Exploiting Self-Recovery and Temperature Awareness}{}
% \cvitemwithcomment{2018}{HeatWatch: Exploiting 3D NAND}{HPCA Lightning Talk \textbf{(on YouTube)}}
% \cvitemwithcomment{}{Self-Recovery and Temperature Effects}{\textbf{\href{https://youtu.be/7ZpGozzEVpY}{https://youtu.be/7ZpGozzEVpY}}}
% \cvitemwithcomment{2016}{Online Flash Channel Modeling and Its Applications}{Flash Memory Summit}
% \cvitemwithcomment{2015}{Data Retention in MLC NAND Flash Memory}{Flash Memory Summit}
% \cvitemwithcomment{2015}{WARM\@: Write-hotness Aware Retention Management}{MSST}
% \cvitemwithcomment{2015}{Read Disturb Errors in MLC NAND Flash Memory}{DSN}
% \cvitemwithcomment{2015}{Data Retention in MLC NAND Flash Memory}{HPCA \textbf{(Best paper session)}}
% \cvitemwithcomment{2014}{Optimizing Data Center Cost via Heterogeneous Reliability Memory}{DSN}


%----------------------------------------------------------------------------------------
%   PROJECTS SECTION
%----------------------------------------------------------------------------------------

\section{Projects}

\subsection{Research Projects}

\ifcameraready%
\cventry{2017.1--2018.3}{Peloton: A Self-Driving In-Memory Database}{Open Source Project}{}{}{
Led a team of three graduate students to design and develop the database catalog for Peloton to support non-blocking schema change and to implement a concurrent lock-free skiplist index. This project is advised by Prof.\ \textbf{\href{http://www.cs.cmu.edu/~pavlo/}{Andy Pavlo}}.
}
\fi%

\cventry{2014.1--2018.7}{Architectural Techniques to Improve NAND Flash Memory Reliability}{}{}{}{
Started as an internship project at Seagate to improve the reliability of NAND flash memory-based SSD at low cost. Led to my PhD dissertation.
\newline{}
% Experimentally characterized state-of-the-art NAND flash memory chips. Developed 4 new analytical models to accurately estimate SSD reliability.
% Developed 10 new techniques to improve SSD lifetime.
% Published 3 first-authored papers and 5 co-authored papers, one of which won the best paper award at DFRWS EU, another won the best paper runner up award at HPCA.
% \newline{}
Detailed achievements:
\begin{itemize}
  \item Published 4 first-authored papers and 6 co-authored papers, one of which won DFRWS EU best paper award, another won HPCA best paper runner up award
  \item Developed 10 new techniques to improve SSD lifetime by up to 12.9$\times$
  \item Experimentally characterized state-of-the-art NAND flash memory chips
  \item Developed 4 new analytical models to accurately estimate SSD reliability
\end{itemize}
}

\cventry{2013.6--2018.7}{Heterogeneous Reliability Memory}{}{}{}{
Started as an internship project at Microsoft Research to optimize data center TCO and memory reliability. Closely related to my PhD dissertation.
\newline{}
% Developed a new server architecture to reduce data center TCO. Developed 
% \newline{}
Detailed achievements:
\begin{itemize}
  \item Published 2 first-authored papers
  \item Developed a new server architecture to reduce data center TCO by 2.7\%
  \item Developed a new mechanism that dynamically adjusts memory capacity and reliability
  % offers multiple levels of error protection, and provides the capacity saved from using weaker protection to applications
\end{itemize}
}

\cventry{2013}{Single-Level Storage}{}{}{}{
Characterized the performance, energy, and scalability benefit of a single-level storage system compared to a traditional two-level storage system through architectural simulations.
% \newline{}
% Detailed achievements:
% \begin{itemize}
%   \item Performed architectural simulations to show the performance, energy, and scalability improvement of a single-level storage system compared to a traditional two-level storage system
% \end{itemize}
}

\cventry{2011.5--2012.3}{A Case for Unlimited Watchpoints}{Undergraduate Research Project}{}{}{
Worked with Prof.\ \textbf{\href{http://web.eecs.umich.edu/~taustin/}{Todd M. Austin}} and Dr.\ \textbf{\href{http://www.computermachines.org/joe/}{Joseph L. Greathouse}} on architecture support for \textbf{Unlimited Watchpoints} that accelerates dynamic software analysis by 9$\times$.
\newline{}
Detailed achievements:
\begin{itemize}
  \item Developed a simulation framework for range cache using C++
  \item Performed architectural simulations to show the performance benefits of the proposed range cache design
\end{itemize}}

\cventry{2011.9--2012.2}{Computational Sprinting}{Undergraduate Research Project}{}{}{
Worked with Prof.\ \textbf{\href{http://web.eecs.umich.edu/~marios/}{Marios C. Papaefthymiou}} and Prof.\ \textbf{\href{http://web.eecs.umich.edu/~twenisch/}{Thomas F. Wenisch}} on \textbf{Computational Sprinting} of manycore processors on mobile devices that improves the responsiveness of interactive applications by 10$\times$.
\newline{}
Detailed achievements:
\begin{itemize}
  \item Developed a SPICE power model for power gating many-core processors
  \item Developed a new technique to reduce the performance overhead for power gating
  \item Performed SPICE circuit simulations to show the performance benefit of Computational Sprinting
\end{itemize}}


\subsection{Academic Projects}

\ifcameraready%
\else%
\cventry{2017.1--2018.3}{Peloton: A Self-Driving In-Memory Database}{Course Project}{}{}{
Developed a system catalog that supports non-blocking schema change and a concurrent lock-free skiplist index.
}
\fi%

\cventry{2017.12}{Multi-Agent Deep Reinforcement Learning}{Course Project}{}{}{
Developed DQN, DDPG, and MADDPG models for continuous multi-agent environment.
}

\cventry{2017.4}{Symbolic Information Processing for Question Answering}{Course Project}{}{}{
Developed an RNN model to answer questions regarding a natural language context.
}

\cventry{2014.12}{Deep Learning with Noise}{Course Project}{}{}{
Characterized the effect of different types of noise on different components of a neural network.
}

\cventry{2014.5}{Compiler Support for Hardware Compression}{Course Project}{}{}{
Developed data splitting and memory pooling compiler optimizations for cache compression algorithms.
}

\cventry{2011.5}{CPU Architecture Design}{Undergraduate Major Design Project}{}{}{
Led a team of three undergraduate students on designing and implementing an 150~MHz out-of-order processor using Verilog.}

\cventry{2011.12}{CPU Layout Design}{Undergraduate Major Design Project}{}{}{
Led a team of five graduate and undergraduate students on designing the circuit layout for a 5-stage pipelined in-order processor and a 3-transistor eDRAM cache.}

%----------------------------------------------------------------------------------------
%   TEACHING SECTION
%----------------------------------------------------------------------------------------
% \newpage
\section{Teaching Experience}

\cventry{2014.8--2012.12}{Teaching Assistant}{Carnegie Mellon University}{Pittsburgh, PA}{}{
CMU 18--742 --- \textbf{Parallel Computer Architecture}, taught by Prof.\ \textbf{\href{http://users.ece.cmu.edu/~omutlu/}{Onur Mutlu}}. \newline{}
Responsibilities include holding office hours, mentoring research projects.}

\cventry{2014.1--2014.5}{Teaching Assistant}{Carnegie Mellon University}{Pittsburgh, PA}{}{
CMU 15--418/15--618 --- \textbf{Parallel Computer Architecture and Programming}, taught by Prof.\ \textbf{\href{https://www.cs.cmu.edu/~kayvonf/}{Kayvon Fatahalian}}. \newline{}
Responsibilities include holding office hours, mentoring projects, preparing and grading homework.}

%----------------------------------------------------------------------------------------
%   PUBLICATIONS SECTION
%----------------------------------------------------------------------------------------

\section{Selected Coursework}

\subsection{Graduate}

\cvitem{CMU 15--721}{Advanced Database Systems}
\cvitem{CMU 15--712}{Advanced Operating Systems and Distributed Systems}
\cvitem{CMU 15--719}{Advanced Cloud Computing}
\cvitem{CMU 15--744}{Computer Networks}
\cvitem{CMU 15--745}{Advanced Optimizing Compilers}
\cvitem{CMU 15--740}{Computer Architecture}
\cvitem{CMU 15--750}{Graduate Algorithms}
\cvitem{CMU 10--701}{Machine Learning}
\cvitem{CMU 10--707}{Deep Learning}
\cvitem{CMU 10--703}{Deep Reinforcement Learning}

\subsection{Undergraduate}

\cvitem{UM EECS570}{Computer Architecture + Major Design Project}
\cvitem{UM EECS427}{VLSI Design + Major Design Project}
\cvitem{UM EECS470}{Microprocessor-Based Systems}
\cvitem{UM EECS482}{Operating Systems}
\cvitem{UM EECS484}{Database Management Systems}
\cvitem{UM EECS492}{Artificial Intelligence}
\cvitem{SJTU}{Honors Mathematics}

%----------------------------------------------------------------------------------------
%   COMPUTER SKILLS SECTION
%----------------------------------------------------------------------------------------

\section{Programming Skills}

\cvitem{Advanced}{C++, Python, Matlab, Shell, Verilog, \LaTeX}
\cvitem{Intermediate}{Perl, HTML, Windows Batch, TensorFlow, PyTorch}
\cvitem{Tools}{Intel Pin, HSPICE, Cadence tools, gem5, Multi2Sim, MySQL/PostgreSQL}

%----------------------------------------------------------------------------------------
%   INTERESTS SECTION
%----------------------------------------------------------------------------------------

\section{Interests}

% \renewcommand{\listitemsymbol}{-~} % Changes the symbol used for lists

\cvlistdoubleitem{Basketball}{Ping Pong}
\cvlistdoubleitem{Hiking}{Traveling}

%----------------------------------------------------------------------------------------

\end{document}